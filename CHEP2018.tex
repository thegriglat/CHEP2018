%%%%%%%%%%%%%%%%%%%%%%% file template.tex %%%%%%%%%%%%%%%%%%%%%%%%%
%
% This is a template file for Web of Conferences Journal
%
% Copy it to a new file with a new name and use it as the basis
% for your article
%
%%%%%%%%%%%%%%%%%%%%%%%%%% EDP Science %%%%%%%%%%%%%%%%%%%%%%%%%%%%
%
%%%\documentclass[option]{webofc}
%%% "twocolumn" for typesetting an article in two columns format (default one column)
%
\documentclass{webofc}
%\documentclass[12pt]{article}
\usepackage[varg]{txfonts}   % Web of Conferences font
%\usepackage{bera}% optional: just to have a nice mono-spaced font
\usepackage{courier}
\usepackage{listings}  % for JSON in appendix
\usepackage{xcolor}
\usepackage{hyperref}
\usepackage[T2A]{fontenc}
\usepackage[utf8]{inputenc}
\usepackage[russian,english]{babel}

\lstset{basicstyle=\normalsize\ttfamily}
%
% Put here some packages required or/and some personnal commands
%
%
\begin{document}
%
\title{Validation web application for detector simulation tools}
%
% subtitle is optionnal
%
%%%\subtitle{Do you have a subtitle?\\ If so, write it here}

\author{\firstname{Luc}~\lastname{Freyermuth}\inst{3}\fnsep\thanks{\email{Luc.Alexandre.Freyermuth@cern.ch}},
        \firstname{Dmitri}~\lastname{Konstantinov}\inst{1}\fnsep\thanks{\email{Dmitri.Konstantinov@cern.ch}},
        \firstname{Grigorii}~\lastname{Latyshev}\inst{2}\fnsep\thanks{\email{Grigorii.Latyshev@cern.ch}}, \newline
        \firstname{Witold}~\lastname{Pokorski}\inst{1}\fnsep\thanks{\email{Witold.Pokorski@cern.ch}} and
        \firstname{Alberto}~\lastname{Ribon}\inst{1}\fnsep\thanks{\email{Alberto.Ribon@cern.ch}}	
        % etc.
}
%\author{L.~Freyermuth \and D.~Konstantinov \and G.~Latyshev \and W.~Pokorski \and A.~Ribon}

\institute{CERN \and IHEP (Protvino, Russia) \and EISTI (Cergy, France)
          }

\abstract{%
One of the key factors for the successful development of physics Monte-Carlo is to properly organize regression testing and validation. Geant4, the world-standard toolkit for HEP detector simulation, heavily relies on this activity. The CERN/SFT group, which contributes to the development, testing, deployment and support of the toolkit, is also in charge of running on a monthly basis a set of community-developed tests using the development releases of Geant4.
We present the Web application geant-val developed for visualizing the results of these tests so that comparisons between different Geant4 releases can be made. The application is written using Express.js, Node.js and Angular frameworks, and uses PostgreSQL for storing test results. Test results are visualized using ROOT and JSROOT. In addition to pure visual comparisons, we perform different statistical tests (chi squared, Kolmogorov-Smirnov, etc) on the client side using Web Workers (JavaScript).

}
%
\maketitle
%
\section{Motivation}
\label{sec-motivation}

Motivation to create the web application. Maybe like more detailed abstract.
Not sure that it is needed.



\section{Software components}
\label{sec-webapplication}

The components of \textsf{Geant-val} web application and interactions between them are shown in the Fig.~\ref{fig:gvp_dataflow}:

\begin{figure}[h]
    \centering
    \includegraphics[width=0.8\textwidth,clip]{schema.png}
    \caption{Components of \textsf{Geant-val} web application and flow of data between them.}
    \label{fig:gvp_dataflow}
\end{figure}

\begin{itemize}

\item \textbf{The server} is the core of the Geant4 validation system. It provides a web API that allows clients to access the database, asynchronously responds to the clients requests and generates high quality plots "on the fly" whenever they are requested.
The server is written in JavaScript and runs with the Node.js engine.

\item \textbf{The database} is used for storing plots containing simulation results or experimental data, together with meta data describing these plots.
%: tool name and version (e.g. \texttt{Geant4, 10.4.p02}), name of the test by which the plot was produced, or Inspire (HepData) ID of the article, etc.
PostgreSQL~\cite{Postgre} is used as database management system for the application. The database instance is provided by the CERN Database-On-Demand service. The database schema is designed in a way to store scatter plots and histograms with unlimited number of optional test parameters in additional to a few mandatory ones.

% db schema

\item A ROOT-based C++ \textbf{plotting utility} was developed to produce high quality plots. It uses data in the JSON format which has been introduced as main interchange format between all parts of the application. The utility is not deeply integrated in the server infrastructure and can be used as a standalone application. It supports all types of application's data, can plot histograms with different binning on one canvas, and produce ratio plots. Ranges and scales of plot axes are selected automatically, but can be overriden if necessary. % User-defined styles are also supported.

\item \textbf{Web interface} is an AngularJS single page application which shows plots with tests results together with statistical analysis. The web page is well working on most popular mobile and desktop browsers (Firefox, Chromium-based browsers, Safari, Edge).

%AngularJS is chosen because it allows to build efficient web applications with highly reusable components.

% The SPA contains 2 pages (see~\ref{sec-analyse} for details):
% \begin{itemize}
%     \item User layouts
%     \item Statistical comparisons
% \end{itemize}

%\begin{figure}[h]
%    \centering
%    \includegraphics[width=0.8\textwidth,clip]{layout_sc.png}
%    \caption{Example of user defined layout for the Geant4 "simplified calorimeter" test showing test results for two Geant4 reference releases - 10.4.ref07 and 10.4.ref08.}
%    \label{fig:layouts}
%\end{figure}

%\textit{User layouts} page (see Fig.~\ref{fig:layouts}). Some Geant4 tests produce hundreds of different plots, but for fast "visual" validation it is often enough to compare only a small well-defined subset of them. The \textit{User layout} is an XML file describing what plots should be displayed and how should they be laid out on a page. User can use one of the existing layouts or define their own one (see Appendix~\ref{adx:XML-format}).
% is used to perform fast visual validation of Geant4.
% One can use one of the existing templates or create and upload their own one.
% It is possible to compare different versions between themselves or against experimental data.
% If two or more Geant4 versions are selected, one can also select a "reference" dataset and plot the ratio of other datasets to it.

%\begin{figure}[h]
%    \centering
%    \includegraphics[width=0.8\textwidth,clip]{statcomparison.png}
%    \caption{Example of statistical comparison between two official Geant4 releases, 10.5.beta01 and 10.4.p02, for the "simplified calorimeter" test.}
%    \label{fig:statcomparison}
%\end{figure}

%\textit{Statistical comparisons} page (see Fig.~\ref{fig:statcomparison}) allows one to perform comparison of simulation with compatible experimental results using statistical tests. It displays results of statistical comparison for pairs of plots with the same parameters' values.
%Currently $\chi^2$ ($\chi^2/n.d.f.$, $\chi^2$ probability) and Kolmogorov-Smirnov (KS Max(D), KS probability) tests are implemented. All computations are performed asynchronously on the client side using JavaScript WebWorkers.
%For this purpose, JavaScript code to perform $\chi^2$ and Kolmogorov-Smirnov tests have been written, and their results cross-checked against the same statistical techniques implemented in the ROOT framework.


%On the \textit{experimental data} page (see Fig.~\ref{fig:exppage}) a summary table of available experimental data is displayed. The data is extracted from original articles or imported from HepDATA~\cite{hepdata} portal. % The experimental data is not linked with particular test and can be used by any tests if parameters match with simulation data.

%\begin{figure}[h]
%    \centering
%    \includegraphics[width=0.8\textwidth,clip]{expdata.png}
%    \caption{Available experimental data.}
%    \label{fig:exppage}
%\end{figure}

Each plot on the website can be displayed as a static image produced by plotting utility or as an interactive JSROOT canvas, which allows changing axes ranges, scales and styles of the data "on the fly" in the same way as it can be done in ROOT. It is possible to export the plots in PNG, ROOT, EPS, Gnuplot~\cite{gnuplot}. It is also possible to view JSON representation of \textit{plot} objects corresponding to the displayed distributions.

Access to the test results produced with internal Geant4 releases is restricted to the Geant4 developers authenticated via CERN Single Sign-On. For public Geant4 releases the access is open to anyone.

%Don't forget to give each section, subsection, subsubsection, and
%paragraph a unique label (see Sect.~\ref{sec-1}).

%For one-column wide figures use syntax of figure~\ref{fig-1}
%\begin{figure}[h]
% Use the relevant command for your figure-insertion program
% to insert the figure file.
%\centering
%\includegraphics[width=1cm,clip]{tiger}
%\caption{Please write your figure caption here}
%\label{fig-1}       % Give a unique label
%\end{figure}

%For two-column wide figures use syntax of figure~\ref{fig-2}
%\begin{figure*}
%\centering
% Use the relevant command for your figure-insertion program
% to insert the figure file. See example above.
% If not, use
%\vspace*{5cm}       % Give the correct figure height in cm
%\caption{Please write your figure caption here}
%\label{fig-2}       % Give a unique label
%\end{figure*}

%For figure with sidecaption legend use syntax of figure
%\begin{figure}
% Use the relevant command for your figure-insertion program
% to insert the figure file.
%\centering
%\sidecaption
%\includegraphics[width=5cm,clip]{tiger}
%\caption{Please write your figure caption here}
%\label{fig-3}       % Give a unique label
%\end{figure}

%For tables use syntax in table~\ref{tab-1}.
%\begin{table}
%\centering
%\caption{Please write your table caption here}
%\label{tab-1}       % Give a unique label
% For LaTeX tables you can use
%\begin{tabular}{lll}
%\hline
%first & second & third  \\\hline
%number & number & number \\
%number & number & number \\\hline
%\end{tabular}
% Or use
%\vspace*{5cm}  % with the correct table height
%\end{table}

% it is needed?
% To manage a set of Geant4 tests and their configurations, a {\tt mc-config-generator} \textbf{Python-based framework} was developed. It allows one to configure and run test jobs in various batch systems (CERN LSF, HTCondor, Torque PBS), and to convert the results into  JSON format for further uploading to the application database. The framework is not Geant4-specific, and can be used with other projects (e.g., Pythia8). Source code is available in corresponding Git repository\footnote{https://gitlab.cern.ch/GeantValidation/geant-config-generator}.

\end{itemize}

\section{Tests}
\label{sec-tests}

At the moment 12 Geant4 tests are integrated in Geant Validation system: FluctTest, FragTest, MscHanson, TestEm3, TestEm9~(Cut, Energy), Hadr00, simplified calorimeter, test15, test22~(HARP, NA49, NA61), test37, test46, tileatlas. These tests are used by CERN Geant4 team for monthly validation.

All listed tests can be configured and run with geant-config-generator utility, which is a part of Geant4 validation system we developed.

\subsection{Geant-config-generator utility}
\label{sec-geant-config-generator}

geant-config-generator is an utility for managing and producing configuration files for tests. It is not Geant4-specific, and can be used for any other software (e.g., Pythia8). It allows to produce test configurations, to run them locally or on various batch systems (currently only CERN LSF and HTCondor supported) and to analyze the results. Source code is available on
CERN Gitlab repository\footnote{https://gitlab.cern.ch/GeantValidation/geant-config-generator}.

Each test integrated in geant-config-generator system contains two parts:

\begin{itemize}
	\item Test configuration files with parameters of the test and template for configuration;
	\item Python class for parsing test results and converting them into the application's JSON format (see Appendix~\ref{adx:JSON-format}).
\end{itemize}

We run tests for all possible combinations of parameters values (the Cartesian product of value sets). For example, for "Hadr00" test it produces 480 combinations of parameters.

For each tests Python parser class is written. Parsing procedure automatically paralleled over all available CPU to gain maximum efficiency and execution speed.

%Python class that parses test output inherits from {\tt geant-config-generator}'s {\tt BaseParser} class in the following way:

%\begin{verbatim}
%from gts.BaseParser import BaseParser
%from gts.utils import getJSON

%class Test(BaseParser):
%    TEST = "test37"
%    IGNOREKEYS = []
%
%    def parse(self, jobs):
%        # your code here
%        yield getJSON(...)
%\end{verbatim}

\subsection{Validation work flow}
\label{sec-workflow}

To perform Geant4 validation using given test one needs to run test, convert test's output to Geant validation JSON format (see Appendix~\ref{adx:JSON-format}) and upload produced files to the website.

To build test it is possible (but not necessary) to use our  repository\footnote{https://gitlab.cern.ch/GeantValidation/geant-validation-tests} which provides a simply way to compile tests for any version of Geant4 (if test supports the given Geant4 version).
Then it is needed to integrate the test in geant-config-generator system (see Sec.~\ref{sec-geant-config-generator}). This step is necessary as geant-config-generator provides an unified way to parse test's result and convert them to JSON files. JSON files uploading is made by our console utility geant\_upload.py. After uploading the results are available on the website to view and analyze.


%\subsection{Geant tests repository}
%\label{sec-geant-validation-tests}

%Sometimes we have to modify original tests developed by Geant4 community to either make their output more suitable for parsing or to fix some unsafe behaviour (like using default physics model). These modified tests are kept in CERN Gitlab repository\footnote{https://gitlab.cern.ch/GeantValidation/geant-validation-tests}.

%We have also developed tools to build all test for a given Geant4 release or releases.
%Also we wrote build system to have a possibility to build tests for all Geant4 releases (if test supports it). In additional {\tt geant-validation-tests} build system encapsulates debug information in each test (compile flags and git hash) to simplify test's debugging.

\section{Plans}
\label{sec-status}

There are major tasks to be done in the future:
\begin{itemize}
	\item Migrate the client side of the application to the Angular2+ framework;
	\item Improve the statistical comparison page;
	\item Improve the {\tt mc-config-generator} framework;
	\item Add more tests for Geant4 as well as for other projects: this is, of course, the most important and (human-)time consuming task!
\end{itemize}

\begin{thebibliography}{}
%
% and use \bibitem to create references.
%
\bibitem{Geant4}
S.~Agostinelli {\it et al.} [GEANT4 Collaboration],
  %``GEANT4: A Simulation toolkit,''
  Nucl.\ Instrum.\ Meth.\ A {\bf 506}, 250 (2003).
  doi:10.1016/S0168-9002(03)01368-8
  %%CITATION = doi:10.1016/S0168-9002(03)01368-8;%%

\bibitem{AngularJS}
AngularJS, v.1.4.8, 20.11.2015, \url{https://github.com/angular/angular.js/commit/75e876424da5f569481488d03cf3a61441341513}

\bibitem{Bootstrap}
Bootstrap, v.3.3.7, 25.07.2016, \url{https://github.com/twbs/bootstrap/commit/0b9c4a4007c44201dce9a6cc1a38407005c26c86}

\bibitem{KaTeX}
KaTeX, v.0.8.3, 28.08.2017,
\url{https://github.com/Khan/KaTeX/commit/dd46d1b00be67b9c859786e134c35495b21ca66b}

\bibitem{JSROOT}
Bertrand Bellenot and Sergey Linev 2015, J. Phys.: Conf. Ser.,664,  062033

\bibitem{ROOT}
    Rene Brun and Fons Rademakers,
    ROOT - An Object Oriented Data Analysis Framework,
    Proceedings AIHENP'96 Workshop, Lausanne, Sep. 1996, Nucl. Inst. & Meth. in Phys. Res. A 389 (1997) 81-86. See also [root.cern.ch/](http://root.cern.ch/).

%\bibitem{RefJ}
% Format for Journal Reference
%Journal Author, Journal \textbf{Volume}, page numbers (year)
% Format for books
%\bibitem{RefB}
%Book Author, \textit{Book title} (Publisher, place, year) page numbers
% etc
\end{thebibliography}


\definecolor{eclipseStrings}{RGB}{42,0.0,255}
\definecolor{eclipseKeywords}{RGB}{127,0,85}
\colorlet{numb}{magenta!60!black}

\lstdefinelanguage{json}{
    basicstyle=\normalfont\ttfamily,
    commentstyle=\color{eclipseStrings}, % style of comment
    stringstyle=\color{eclipseKeywords}, % style of strings
    numbers=left,
    numberstyle=\scriptsize,
    stepnumber=1,
    numbersep=8pt,
    showstringspaces=false,
    breaklines=true,
%    frame=lines,
    backgroundcolor=\color{white}, %only if you like
}

\begin{figure}

\begin{lstlisting}[language=json,firstnumber=1]
{
  "article": {"inspireId": -1},
  "mctool": {"name": "Geant4", "version": "", "model": ""},
  "testName": "",
  "metadata": {"observableName": "", "reaction": "",
    "targetName": "", "beamParticle": "",
    "beamEnergies": [], "secondaryParticle": "",
    "parameters": [
      {
        "names": "THETA",
        "values": "60 degrees"
      }
    ]
  },
  // for scatter data
  "plotType": "SCATTER2D",
  "chart": {
    "nPoints: 0,
    "title": "", "xAxisName": "", "yAxisName": "",
    "xValues": [], "yValues": [],
    "xStatErrorsPlus": [], "xStatErrorsMinus": [],
    "yStatErrorsPlus": [], "yStatErrorsMinus": [],
    "xSysErrorsPlus": [], "xSysErrorsMinus": [],
    "ySysErrorsPlus": [], "ySysErrorsMinus": []
  },
  // for histogram data
  "plotType": "TH1",
  "histogram": {
    "nBins: [0],
    "title: "", "xAxisName": "", "yAxisName": "",
    "binEdgeLow": [], "binEdgeHigh": [], "binContent": [],
    "yStatErrorsPlus": [], "yStatErrorsMinus": [],
    "ySysErrorsPlus": [], "ySysErrorsMinus": [],
    "binLabel": []
  }
}
\end{lstlisting}

\caption{Example of JSON format used by web application.}
\label{JSON-format}
\end{figure}

\begin{figure}
\begin{verbatim}
/random/setSavingFlag 1
/run/verbose 1
/event/verbose 0
/tracking/verbose 0
/gun/particle %PARTICLE%
/gun/energy %ENERGY% %ENERGY_UNIT%
%DETECTOR%
/run/beamOn %NEVENTS%
\end{verbatim}
\caption{Template file for {\tt simplified calorimeter} test.}
\label{sc-template}
\end{figure}

\begin{figure}
\begin{verbatim}
!PARTICLE=pi-
!PHYSLIST=FTFP_BERT, FTFP_BERT_HP, QGSP_BERT, QGSP_BERT_HP,  FTFP_BERT_TRV,
FTFP_BERT_ATL, QGSP_INCLXX, QGSP_BIC, QGSP_FTFP_BERT
!DETECTOR=include AtlasECAL, include AtlasFCAL, include AtlasHEC,
include CmsECAL, include LhcbECAL, include TileCal
!CONST:RANDOMSEED=AUTO
!CONST:ENERGY_UNIT=GeV
   PARTICLE  | ENERGY | PHYSLIST | DETECTOR | NEVENTS
2* PARTICLE  | 1.     | PHYSLIST | DETECTOR | 2500
2* PARTICLE  | 2.     | PHYSLIST | DETECTOR | 2500
2* PARTICLE  | 3.     | PHYSLIST | DETECTOR | 2500
2* PARTICLE  | 4.     | PHYSLIST | DETECTOR | 2500
2* PARTICLE  | 5.     | PHYSLIST | DETECTOR | 2500
2* PARTICLE  | 6.     | PHYSLIST | DETECTOR | 2500
2* PARTICLE  | 7.     | PHYSLIST | DETECTOR | 2500
2* PARTICLE  | 8.     | PHYSLIST | DETECTOR | 2500
2* PARTICLE  | 9.     | PHYSLIST | DETECTOR | 2500
2* PARTICLE  | 10.    | PHYSLIST | DETECTOR | 2500
2* PARTICLE  | 11.    | PHYSLIST | DETECTOR | 2500
2* PARTICLE  | 12.    | PHYSLIST | DETECTOR | 2500
2* PARTICLE  | 13.    | PHYSLIST | DETECTOR | 2500
2* PARTICLE  | 14.    | PHYSLIST | DETECTOR | 2500
2* PARTICLE  | 15.    | PHYSLIST | DETECTOR | 2500
2* PARTICLE  | 16.    | PHYSLIST | DETECTOR | 2500
2* PARTICLE  | 17.    | PHYSLIST | DETECTOR | 2500
2* PARTICLE  | 18.    | PHYSLIST | DETECTOR | 2500
2* PARTICLE  | 19.    | PHYSLIST | DETECTOR | 2500
5* PARTICLE  | 20.    | PHYSLIST | DETECTOR | 1000
5* PARTICLE  | 25.    | PHYSLIST | DETECTOR | 1000
5* PARTICLE  | 50.    | PHYSLIST | DETECTOR | 1000
20* PARTICLE | 100.   | PHYSLIST | DETECTOR | 125
20* PARTICLE | 200.   | PHYSLIST | DETECTOR | 125
20* PARTICLE | 500.   | PHYSLIST | DETECTOR | 125
\end{verbatim}
\caption{Parameter's file for {\tt simplified calorimeter} test.}
\label{sc-parameters}
\end{figure}


\begin{figure}

\begin{verbatim}
#!/bin/bash
source config.sh
export PHYSLIST="%PHYSLIST%"
export SHOWER_MAP_OFF="1"
export G4NEUTRONHP_NEGLECT_DOPPLER="1"
StatAccepTest sc.mac %RANDOMSEED% 1>test_stdout.txt 2>test_stderr.txt
\end{verbatim}
\caption{Run file for {\tt simplified calorimeter} test.}
\label{sc-run}
\end{figure}


\end{document}

% end of file template.tex

%<div id='footer'><table width='100%'><tr><td class='right'><a href='http://fusioninventory.org/'><span class='copyright'>FusionInventory 9.1+1.0 | copyleft <img src='/glpi/plugins/fusioninventory/pics/copyleft.png'/>  2010-2016 by FusionInventory Team</span></a></td></tr></table></div>