% \section{Performing validation with Geant-val}
% \label{sec-workflow}

% Geant4 validation procedure consists of two big steps: generating results (executing the tests and importing their output into Geant-val) and analysing these results.

% To perform the validation of Geant4 using \textsf{Geant-val} one needs to run all the tests, convert their output into the JSON format and upload the produced files to the website.

\section{Generating results}

\subsection{Running the tests}

To manage a set of Geant4 tests and their configurations, a {\tt mc-config-generator} Python framework was developed. It allows one to configure and run test jobs in various batch systems (CERN LSF, HTCondor, Torque PBS), and to convert the results into  JSON format for further uploading to the application database. The framework is not Geant4-specific, and can be used with other projects (e.g., Pythia8). Source code is available in corresponding Git repository\footnote{https://gitlab.cern.ch/GeantValidation/geant-config-generator}.

Each test supported by the {\tt mc-config-generator} consists of a "parameters" file, containing all combinations of parameters used by the given test, and one or more "template" files.

\subsection{Importing test results into Geant-val}

We use JSON, like shown in Appendix~\ref{adx:JSON-format}, for importing data into the database and for exchanging information between different parts of the web application. Each JSON file contains one plot scatter plot or 1D histogram, along with additional data:

\begin{itemize}
    \item Name of the measured physics value (\textit{observable}), which used for matching \textit{plot} to the experimental data (e.g., differential cross-section);
    \item The name and version of tool (e.g., Geant4 and 10.5.beta01);
    \item Name of the test (e.g. Tileatlas);
    \item Name(s) and value(s) of the parameters used for running the test (e.g. Physics List: FTFP\_BERT);
    \item For experimental data, Inspire~\cite{inspire} or HepDATA~\cite{hepdata} ID of the original article.
\end{itemize}

For the supported tests, the parser script in {\tt mc-config-generator} framework can be used to convert text and/or ROOT files produced by the test into JSON format. The conversion is done in multiple threads to minimise the execution time.

Geant4 developers can upload JSON files by sending a HTTP POST request to \textsf{Geant-val} website using {\tt geant\_upload.py}\footnote{https://gitlab.cern.ch/GeantValidation/GVP/raw/master/scripts/geant\_upload.py} command-line utility.

After uploading the results, these are available on the website for further analysis.

