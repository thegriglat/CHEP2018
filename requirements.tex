\section{Requirements}
\label{sec:requirements}

Software validation is a continuous and complex process involving the entire Geant4 community. The main goal of the application we describe here is to provide a fast and convenient web-based validation tool for Geant4 developers.

Below there is a brief list of the requirements to the validation application:
%that were trying to implement in this validation application
% and how they were implemented:
\begin{itemize}
    \item The procedure for integrating new Geant4 tests should be simple and straightforward;
    \item User should be able to display and analyse one- and multidimensional histograms and scatter plots, with same or different binning;
    \item User should be able to compare tests results with experimental data, when available;
    \item It should be possible to produce high quality plots with custom plot style, ranges, scales, etc using only the website;
    % \item Providing to users as much information about tests results as possible;
    % \item High parallelism to run and analyse a large set of Geant4 tests in reasonable time; % To achieve this the system should provide a way to run Geant4 tests in highly diversified jobs in batch systems.
    \item The application should have intuitive, modern and fast user interface;
    %% \item Scalability:
    \item Should use the technologies that allows a fast feature development and integration.
\end{itemize}

In order to keep data consistent between all application services a \textit{plot} data type was introduced. A \textit{plot} object is a central element of the data model of \textsf{Geant-val} application, used for exchanging data between different parts of the web application: displaying data on the web page, calculating result of statistical tests, etc. Files containing \textit{plot} objects in JSON~\cite{json} format are used to upload data to the database and to generate plots with standalone C++ plotter utility.
Each \textit{plot} consists of histogram or scatter plot data and necessary metadata describing the configuration used to run the associated Monte-Carlo simulation (Geant4 version, physics list), test parameters (beams, targets, energies, etc.) and its results (e.g., the name of the quantity measured).

A Python tool named {\tt mc-config-generator} was developed to produce JSON files containing \textit{plot} objects from the output of the validation application. It also provides a uniform way to prepare and run tests on the batch system (LSF, Torque PBS, HTCondor), and to get information about execution times and any Geant4 exceptions generated during test execution.

For faster deployment, the application is distributed as a Docker image which is hosted on virtual machine provided by CERN VM service. The website is optimised for asynchronous access and plotting and can be easily scaled ''horizontally'' by launching new instances.%, as the application has no software and hardware requirements except presence of running Docker server.