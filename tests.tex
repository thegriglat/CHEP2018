To manage a set of Geant4 tests and their configurations Python tool geant-config-generator was developed. The tool allows to configure and run test's jobs and parse simulation results to JSON format for further uploading to the application database.

geant-config-generator is an utility for managing and producing configuration files for tests. It is not Geant4-specific, and can be used for any other software (e.g., Pythia8). It allows to produce test configurations, to run them locally or on various batch systems (currently only CERN LSF and HTCondor supported) and to analyze the results. Source code is available on
CERN Gitlab repository\footnote{https://gitlab.cern.ch/GeantValidation/geant-config-generator}.

To be used in Geant validation application two test related parts should be added in geant-config-generator:

\begin{itemize}
	\item Test configuration files with parameters of the test and template for configuration;
	\item Python class for parsing test results and converting them into the application's JSON format (see Appendix~\ref{adx:JSON-format}).
\end{itemize}

According to these files geant-config-generator produces Cartesian product of input parameters and store prepared configuration files in "jobs". Later the jobs can be submitted in batch system (CERN LSF or HTCondor) or ran locally. In additional geant-config-generator provides friendly interface to monitor status of jobs and measure execution time for each job.

The Python parser class should be written for test to convert test's result to the application JSON format. Further these JSON files can be uploaded to the site to be available for results comparison.
Parse procedure automatically paralleled over all available CPU to gain maximum execution speed.
Python parse class inherits from $BaseParser$ class as listed in example below:

\begin{lstlisting}[language=python,firstnumber=1]
#!/usr/bin/env python
from gts.BaseParser import BaseParser
from gts.utils import getJSON

class Test(BaseParser):
    TEST = "TestName"
    IGNOREKEYS = []

    def parse(self, jobs):
        # your code here
        yield getJSON(...)
\end{lstlisting}

\section{Current status}
\label{sec-status}

The validation application and geant-config-generator utility are ready to use by Geant4 developers. The application's code is passed CERN IT security control.

At the moment 12 Geant4 tests are integrated in Geant Validation system: FluctTest, FragTest, MscHanson, TestEm3, TestEm9, Hadr00, simplified calorimeter, test15, test22~(HARP, NA49, NA61), test37, test46, tileatlas. For all these tests predefined layouts templates are available.

The entire application is distributed using Docker technology and can be easily deployed on any operating system.

\section{Validation work flow}
\label{sec-workflow}

To perform Geant4 validation using given test one needs to run test, convert test's output to Geant validation JSON format (see Appendix~\ref{adx:JSON-format}) and upload produced files to the website.

To build a test one can use our  repository\footnote{https://gitlab.cern.ch/GeantValidation/geant-validation-tests}, which provides a simple way to compile a test for any Geant4 version (if this test supports that version of Geant4).
Then one needs to integrate the test in geant-config-generator system. This step is necessary as geant-config-generator provides an unified way to parse test's result and convert them to JSON files. 

Uploading of JSON files is performed by our console utility
\href{https://gitlab.cern.ch/GeantValidation/GVP/raw/master/scripts/geant_upload.py}{geant\_upload.py}.
After uploading the results are available on the website to view and analyze.


%\subsection{Geant tests repository}
%\label{sec-geant-validation-tests}

%Sometimes we have to modify original tests developed by Geant4 community to either make their output more suitable for parsing or to fix some unsafe behaviour (like using default physics model). These modified tests are kept in CERN Gitlab repository\footnote{https://gitlab.cern.ch/GeantValidation/geant-validation-tests}.

%We have also developed tools to build all test for a given Geant4 release or releases.
%Also we wrote build system to have a possibility to build tests for all Geant4 releases (if test supports it). In additional {\tt geant-validation-tests} build system encapsulates debug information in each test (compile flags and git hash) to simplify test's debugging.