To manage a set of Geant4 tests and their configurations Python framework {\tt geant-config-generator} is developed. It allows to configure and run test jobs in batch system (CERN LSF, HTCondor, Torque PBS), and parse simulation results to JSON format for further uploading to the application database.
The framework is not Geant4-specific, and can be used with other projects (e.g., Pythia8). Source code is available on
CERN Gitlab repository\footnote{https://gitlab.cern.ch/GeantValidation/geant-config-generator}.

To be used in Geant validation application two test related parts should be added in {\tt geant-config-generator} system:

\begin{itemize}
	\item Parameters description file and templates for test run;
	\item Python class for converting test output to the application's JSON format (see Appendix~\ref{adx:JSON-format}).
\end{itemize}

According to test's parameters and template files {\tt geant-config-generator} generates Cartesian product of input parameters and store prepared configurations in "jobs". Later the jobs can be submitted in batch system or run locally. In additional {\tt geant-config-generator} provides console commands to monitor status of jobs and measure execution time for each job.

The Python parser class should be written for test to convert test's results to the application JSON files. Further these JSON files can be uploaded to the website to be available for Geant4 validation.
Convert procedure is automatically paralleled over all available CPU to gain maximum execution speed.
% Python parse class inherits from $BaseParser$ class as listed in example below:

% \begin{lstlisting}[language=python,firstnumber=1]
% #!/usr/bin/env python
% from gts.BaseParser import BaseParser
% from gts.utils import getJSON

% class Test(BaseParser):
%     TEST = "TestName"
%     IGNOREKEYS = []

%     def parse(self, jobs):
%         # your code here
%         yield getJSON(...)
% \end{lstlisting}

\section{Current status}
\label{sec-status}

The web application and {\tt geant-config-generator} framework are ready to use by Geant4 test developers. The website is passed security audit by CERN IT department.

At the moment 12 Geant4 tests are fully integrated in Geant Validation system and have layouts on the website (see Table~\ref{table:tests}):

\begin{table}[h]
\centering
\begin{tabular}{lll}
\hline
Author & Tests  \\\hline
Vladimir Ivanchenko & FluctTest, MscHanson, TestEm3, TestEm9, Hadr00, \\ 
& Test22~(HARP, NA49, NA61), Test37, Test46, Tileatlas \\
Susanna Guatelli, David Bolst & FragTest \\
Andrea Dotti, Alberto Ribon & Simplified calorimeter \\
Alexander Howard & Test15 \\

\end{tabular}
\caption{Tests integrated in Geant Validation website.}
\label{table:tests}
\end{table}

%\begin{itemize}
%    \item FluctTest (Vladimir Ivanchenko)
%    \item FragTest (Susanna Guatelli, David Bolst)
%    \item MscHanson (Vladimir Ivanchenko)
%    \item TestEm3 (Vladimir Ivanchenko)
%    \item TestEm9 (Vladimir Ivanchenko)
%    \item Hadr00 (Vladimir Ivanchenko)
%    \item Simplified calorimeter (Andrea Dotti, Alberto Ribon)
%    \item Test15 (Alexander Howard)
%    \item Test22~(HARP, NA49, NA61) (Vladimir Ivanchenko)
%    \item Test37 (Vladimir Ivanchenko)
%    \item Test46 (Vladimir Ivanchenko)
%    \item Tileatlas (Vladimir Ivanchenko)
%\end{itemize}
%FluctTest, FragTest, MscHanson, TestEm3, TestEm9, Hadr00, simplified calorimeter, test15, test22~(HARP, NA49, NA61), test37, test46, tileatlas. 

Currently the application's database contains 86213 tests histograms and charts as well as 1414 experimental datasets.

The web application is distributed using Docker technology and can be easily deployed on any operating system.

\section{Validation work flow}
\label{sec-workflow}

To perform Geant4 validation using given test one needs to run test, convert test's output to Geant validation JSON format (see Appendix~\ref{adx:JSON-format}) and upload produced files to the website.

To build a test it is possible to use CERN Gitlab repository\footnote{https://gitlab.cern.ch/GeantValidation/geant-validation-tests}, which provides a simple way to compile the test for any Geant4 version (if the test supports requested version of Geant4).
Then one needs to integrate the test in {\tt geant-config-generator} system to be able produce and parse tests results.

Uploading of JSON files is performed by {\tt geant\_upload.py}\footnote{https://gitlab.cern.ch/GeantValidation/GVP/raw/master/scripts/geant\_upload.py} console utility.
After uploading the results are available on the website to view and analyze.


%\subsection{Geant tests repository}
%\label{sec-geant-validation-tests}

%Sometimes we have to modify original tests developed by Geant4 community to either make their output more suitable for parsing or to fix some unsafe behaviour (like using default physics model). These modified tests are kept in CERN Gitlab repository\footnote{https://gitlab.cern.ch/GeantValidation/geant-validation-tests}.

%We have also developed tools to build all test for a given Geant4 release or releases.
%Also we wrote build system to have a possibility to build tests for all Geant4 releases (if test supports it). In additional {\tt geant-validation-tests} build system encapsulates debug information in each test (compile flags and git hash) to simplify test's debugging.