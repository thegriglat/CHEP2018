To manage a set of Geant4 tests and their configurations Python framework {\tt mc-config-generator} is developed. It allows to configure and run test jobs in batch system (CERN LSF, HTCondor, Torque PBS), and parse simulation results to JSON format for further uploading to the application database.
The framework is not Geant4-specific, and can be used with other projects (e.g., Pythia8). Source code is available on
CERN Gitlab repository\footnote{https://gitlab.cern.ch/GeantValidation/geant-config-generator}.

To be used in Geant validation application two test related parts should be added in {\tt mc-config-generator} system:

\begin{itemize}
	\item Parameters description file and templates for test run;
	\item Python class for converting test output to the application's JSON format (see Appendix~\ref{adx:JSON-format}).
\end{itemize}

According to test's parameters and template files {\tt mc-config-generator} generates Cartesian product of input parameters and store prepared configurations in "jobs". Later the jobs can be submitted in batch system or run locally. In additional {\tt mc-config-generator} provides console commands to monitor status of jobs and measure execution time for each job.

The Python parser class should be written for test to convert test's results to the application JSON files. Further these JSON files can be uploaded to the website to be available for Geant4 validation.
Convert procedure is automatically paralleled over all available CPU to gain maximum execution speed.
% Python parse class inherits from $BaseParser$ class as listed in example below:

% \begin{lstlisting}[language=python,firstnumber=1]
% #!/usr/bin/env python
% from gts.BaseParser import BaseParser
% from gts.utils import getJSON

% class Test(BaseParser):
%     TEST = "TestName"
%     IGNOREKEYS = []

%     def parse(self, jobs):
%         # your code here
%         yield getJSON(...)
% \end{lstlisting}



%\subsection{Geant tests repository}
%\label{sec-geant-validation-tests}

%Sometimes we have to modify original tests developed by Geant4 community to either make their output more suitable for parsing or to fix some unsafe behaviour (like using default physics model). These modified tests are kept in CERN Gitlab repository\footnote{https://gitlab.cern.ch/GeantValidation/geant-validation-tests}.

%We have also developed tools to build all test for a given Geant4 release or releases.
%Also we wrote build system to have a possibility to build tests for all Geant4 releases (if test supports it). In additional {\tt geant-validation-tests} build system encapsulates debug information in each test (compile flags and git hash) to simplify test's debugging.