
\definecolor{eclipseStrings}{RGB}{42,0.0,255}
\definecolor{eclipseKeywords}{RGB}{127,0,85}
\colorlet{numb}{magenta!60!black}

\lstdefinelanguage{json}{
    basicstyle=\normalsize\ttfamily,
    commentstyle=\color{eclipseStrings}, % style of comment
    stringstyle=\color{eclipseKeywords}, % style of strings
    numbers=left,
    numberstyle=\scriptsize,
    stepnumber=1,
    numbersep=8pt,
    showstringspaces=false,
    breaklines=true,
%    frame=lines,
    backgroundcolor=\color{white}, %only if you like
}

\begin{figure}

\begin{lstlisting}[language=json,firstnumber=1]
{
  "article": {"inspireId": -1},
  "mctool": {"name": "Geant4", "version": "", "model": ""},
  "testName": "",
  "metadata": {"observableName": "", "reaction": "",
    "targetName": "", "beamParticle": "",
    "beamEnergies": [], "secondaryParticle": "",
    "parameters": [
      {
        "names": "THETA",
        "values": "60 degrees"
      }
    ]
  },
  // for scatter data
  "plotType": "SCATTER2D",
  "chart": {
    "nPoints: 0,
    "title": "", "xAxisName": "", "yAxisName": "",
    "xValues": [], "yValues": [],
    "xStatErrorsPlus": [], "xStatErrorsMinus": [],
    "yStatErrorsPlus": [], "yStatErrorsMinus": [],
    "xSysErrorsPlus": [], "xSysErrorsMinus": [],
    "ySysErrorsPlus": [], "ySysErrorsMinus": []
  },
  // for histogram data
  "plotType": "TH1",
  "histogram": {
    "nBins: [0],
    "title: "", "xAxisName": "", "yAxisName": "",
    "binEdgeLow": [], "binEdgeHigh": [], "binContent": [],
    "yStatErrorsPlus": [], "yStatErrorsMinus": [],
    "ySysErrorsPlus": [], "ySysErrorsMinus": [],
    "binLabel": []
  }
}
\end{lstlisting}

\caption{Example of JSON format used by web application.}
\label{JSON-format}
\end{figure}

\begin{figure}
\begin{lstlisting}
/control/verbose 1
/run/verbose 1
/tracking/verbose 0
/testhadr/TargetMat G4_%TARGETELM%
/testhadr/TargetRadius 2 cm
/testhadr/TargetLength  50 cm
/run/printProgress 10
/run/initialize
/process/em/workerVerbose 0
/run/setCut 1 km
/gun/particle proton
/gun/energy 20. GeV
/testhadr/targetElm %TARGETELM%
/testhadr/particle %PARTICLE%
/testhadr/fileName test
/testhadr/verbose 1
/run/beamOn 1
\end{lstlisting}
\caption{Template file for "Hadr00" test.}
\label{hadr00-template}
\end{figure}

\begin{figure}
\begin{lstlisting}
!PARTICLE=proton, e-, kaon-, kaon+, neutron, pi-, pi+, gamma
!TARGETELM=Be, C, Cu, Fe, H, Pb, Si, U, He, Al
!PHYSICS_LIST=FTFP_BERT, QGSP_BERT, FTFP_BERT_HP, FTFP_BERT_TRV, QGSP_BIC, QBBC
PARTICLE | TARGETELM | PHYSICS_LIST
PARTICLE | TARGETELM | PHYSICS_LIST
\end{lstlisting}
\caption{Parameter's file for "Hadr00" test.}
\label{hadr00-parameters}
\end{figure}


\begin{figure}

\begin{verbatim}
#!/bin/bash
source config.sh
export PHYSLIST="%PHYSICS_LIST%"
Hadr00 Hadr00.mac 1>test_stdout.txt 2>test_stderr.txt
\end{verbatim}
\caption{Run file for "Hadr00" test.}
\label{hadr00-run}
\end{figure}