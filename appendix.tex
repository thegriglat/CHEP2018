% reset numbering and title
\setcounter{figure}{0}
\renewcommand{\figurename}{Appendix}

\newpage
\section{Appendix}

\begin{figure}[h]

\begin{lstlisting}[language=json,firstnumber=1]
{
  "article": {"inspireId": -1},
  "mctool": {"name": "Geant4", "version": "", "model": ""},
  "testName": "",
  "metadata": {"observableName": "", "reaction": "",
    "targetName": "", "beamParticle": "",
    "beamEnergies": [], "secondaryParticle": "",
    "parameters": [
      {
        "names": "THETA",
        "values": "60 degrees"
      }
    ]
  },
  // for scatter data
  "plotType": "SCATTER2D",
  "chart": {
    "nPoints: 0,
    "title": "", "xAxisName": "", "yAxisName": "",
    "xValues": [], "yValues": [],
    "xStatErrorsPlus": [], "xStatErrorsMinus": [],
    "yStatErrorsPlus": [], "yStatErrorsMinus": [],
    "xSysErrorsPlus": [], "xSysErrorsMinus": [],
    "ySysErrorsPlus": [], "ySysErrorsMinus": []
  },
  // for histogram data
  "plotType": "TH1",
  "histogram": {
    "nBins: [0],
    "title: "", "xAxisName": "", "yAxisName": "",
    "binEdgeLow": [], "binEdgeHigh": [], "binContent": [],
    "yStatErrorsPlus": [], "yStatErrorsMinus": [],
    "ySysErrorsPlus": [], "ySysErrorsMinus": [],
    "binLabel": []
  }
}
\end{lstlisting}

\caption{Example of JSON format used by the web application.}
\label{adx:JSON-format}
\end{figure}

\begin{figure}[h]

\begin{lstlisting}[language=xml,firstnumber=1,breaklines=true,
  postbreak=\mbox{\textcolor{red}{$\hookrightarrow$}\space}]
<?xml version="1.0" encoding="UTF-8"?>
<layout>
  <default model="FTFP_BERT"></default>
  <row>
    <label text="\LARGE{tileatlas}" colspan="4"/>
  </row>
  <!-- energy response, normalized width, longitudinal shower shape and lateral shower shape -->
  <row>
    <plot test="tileatlas" observable="relative energy deposition" beam="e-" energy="10"   secondary="None" target="tileatlas" parname="THETA" parvalue="0 degrees" yaxis="lin" xaxis="lin" title=""/>
    <plot test="tileatlas" observable="relative energy deposition" beam="e-" energy="10"   secondary="None" target="tileatlas" parname="THETA" parvalue="20 degrees" yaxis="lin" xaxis="lin" title=""/>
  </row>
  <row>
    <plot test="tileatlas" observable="relative energy deposition" beam="e-" energy="20"   secondary="None" target="tileatlas" parname="THETA" parvalue="21 degrees" yaxis="lin" xaxis="lin" title=""/>
    <plot test="tileatlas" observable="relative energy deposition" beam="e-" energy="20"   secondary="None" target="tileatlas" parname="THETA" parvalue="61 degrees" yaxis="lin" xaxis="lin" title=""/>
   </row>
  <row>
    <plot test="tileatlas" observable="relative energy deposition" beam="e-" energy="100"   secondary="None" target="tileatlas" parname="THETA" parvalue="0 degrees" yaxis="lin" xaxis="lin" title=""/>
    <plot test="tileatlas" observable="relative energy deposition" beam="e-" energy="100"   secondary="None" target="tileatlas" parname="THETA" parvalue="10 degrees" yaxis="lin" xaxis="lin" title=""/>
    <plot test="tileatlas" observable="relative energy deposition" beam="e-" energy="100"   secondary="None" target="tileatlas" parname="THETA" parvalue="20 degrees" yaxis="lin" xaxis="lin" title=""/>
  </row>
</layout>
\end{lstlisting}

\caption{Example of User layout XML file.}
\label{adx:XML-format}
\end{figure}

%\begin{figure}
%\begin{lstlisting}
%/control/verbose 1
%/run/verbose 1
%/tracking/verbose 0
%/testhadr/TargetMat G4_%TARGETELM%
%/testhadr/TargetRadius 2 cm
%/testhadr/TargetLength  50 cm
%/run/printProgress 10
%/run/initialize
%/process/em/workerVerbose 0
%/run/setCut 1 km
%/gun/particle proton
%/gun/energy 20. GeV
%/testhadr/targetElm %TARGETELM%
%/testhadr/particle %PARTICLE%
%/testhadr/fileName test
%/testhadr/verbose 1
%/run/beamOn 1
%\end{lstlisting}
%\caption{Template file for "Hadr00" test.}
%\label{hadr00-template}
%\end{figure}

%\begin{figure}
%\begin{lstlisting}
%!PARTICLE=proton, e-, kaon-, kaon+, neutron, pi-, pi+, gamma
%!TARGETELM=Be, C, Cu, Fe, H, Pb, Si, U, He, Al
%!PHYSICS_LIST=FTFP_BERT, QGSP_BERT, FTFP_BERT_HP, FTFP_BERT_TRV, QGSP_BIC, QBBC
%PARTICLE | TARGETELM | PHYSICS_LIST
%PARTICLE | TARGETELM | PHYSICS_LIST
%\end{lstlisting}
%\caption{Parameter's file for "Hadr00" test.}
%\label{hadr00-parameters}
%\end{figure}


%\begin{figure}
%
%\begin{verbatim}
%#!/bin/bash
%source config.sh
%export PHYSLIST="%PHYSICS_LIST%"
%Hadr00 Hadr00.mac 1>test_stdout.txt 2>test_stderr.txt
%\end{verbatim}
%\caption{Run file for "Hadr00" test.}
%\label{hadr00-run}
%\end{figure}