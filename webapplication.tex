\section{Web application}
\label{sec-webapplication}

\subsection{Server side/Back end}
\label{sec-webapplication-server}

The server is the core of the Geant validation system.
%It provides Web API for the clients to request data from PostgreSQL database and generates high quality plots on the fly when they are requested.
It is written in JavaScript using Node.js as execution platform. PostgreSQL DBMS provided by CERN Database On Demand (DBOD) service is used to store all application data. Database schema is designed in way to allow store scatter plots and histograms with unlimited number of parameters effectively. To produce high quality plots we wrote ROOT based C++ utility {\tt plotter} which uses data in out JSON format.

\subsection{Client side/Front end}
\label{sec-webapplication-client}

Client side of application is AngularJS (1.4.8) single page application. Website contains 4 parts:

\begin{itemize}
    \item User layouts
    \item Stat comparison
    \item Experimental data
    \item Page to search values in application's database
\end{itemize}

User layout's page is designed for visual comparison of set of plots. It is allowed to compare different versions as well as against experimental data. Also users can write own templates and use it on layouts page. Each plot on the page can be viewed in static and interactive mode. In case of two or more versions selected one can plot ratio.

Stat comparison page provides possibility to perform statistical validation of Geant4 and experiments for given test. It displays pairs of plots with the same parameters differs only by version. For these pairs results of $\chi^2$ and Kolmogorov-Smirnov tests are displayed. All computations are performed on the client side in the background in separate processes using WebWorkers API. $\chi^2$ and Kolmogorov-Smirnov tests are validated and give the same results as ROOT framework.

On the experiment data page one can see all experimental data loaded in application database.
Lookup table page contains all predefined values in the database such as project's names, versions, observables, particles and other.

Each plot in the application can be viewed in interactive manner via JSROOT~\cite{JSROOT} which allows to change on the fly axes ranges, scales and styles of the data in the same way as it can be done in ROOT. To use plots in his reports/papers the application provides plots in PNG, ROOT and EPS formats. For testing purposes we also provide JSON and Gnuplot output.

In the application there is authorization form which is connected with CERN Single-Sign-On service. Logged in users is able to see and compare testing releases of Geant4.

%Don't forget to give each section, subsection, subsubsection, and
%paragraph a unique label (see Sect.~\ref{sec-1}).

%For one-column wide figures use syntax of figure~\ref{fig-1}
%\begin{figure}[h]
% Use the relevant command for your figure-insertion program
% to insert the figure file.
%\centering
%\includegraphics[width=1cm,clip]{tiger}
%\caption{Please write your figure caption here}
%\label{fig-1}       % Give a unique label
%\end{figure}

%For two-column wide figures use syntax of figure~\ref{fig-2}
%\begin{figure*}
%\centering
% Use the relevant command for your figure-insertion program
% to insert the figure file. See example above.
% If not, use
%\vspace*{5cm}       % Give the correct figure height in cm
%\caption{Please write your figure caption here}
%\label{fig-2}       % Give a unique label
%\end{figure*}

%For figure with sidecaption legend use syntax of figure
%\begin{figure}
% Use the relevant command for your figure-insertion program
% to insert the figure file.
%\centering
%\sidecaption
%\includegraphics[width=5cm,clip]{tiger}
%\caption{Please write your figure caption here}
%\label{fig-3}       % Give a unique label
%\end{figure}

%For tables use syntax in table~\ref{tab-1}.
%\begin{table}
%\centering
%\caption{Please write your table caption here}
%\label{tab-1}       % Give a unique label
% For LaTeX tables you can use
%\begin{tabular}{lll}
%\hline
%first & second & third  \\\hline
%number & number & number \\
%number & number & number \\\hline
%\end{tabular}
% Or use
%\vspace*{5cm}  % with the correct table height
%\end{table}