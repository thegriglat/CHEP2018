\section{Web application}
\label{sec-webapplication}

\subsection{Server side/Back end}
\label{sec-webapplication-server}

The server is the core of the geant-val system.
%It provides Web API for the clients to request data from PostgreSQL database and generates high quality plots on the fly when they are requested.
It is written in JavaScript using Node.js as execution platform. PostgreSQL DBMS provided by CERN Database On Demand (DBoD) service is used to store all application data. Database schema is designed in way to allow store scatter plots and histograms with unlimited number of parameters effectively. To produce high quality plots we wrote ROOT based C++ utility {\tt plotter} which uses data in out JSON format. On the website plots are available in  ROOT, PNG and EPS formats. It is also possible to fetch chart/histogram raw data as JSON or Gnuplot files.

\subsection{Client side/Front end}
\label{sec-webapplication-client}

Client side of application is AngularJS (1.4.8) single page application. Website contains 4 parts:

\begin{itemize}
    \item User layouts
    \item Stat comparison
    \item Experimental data
    \item Page to search values in application's database
\end{itemize}

Layout's page is designed for visual comparison of set of plots. It is allowed to compare different versions as well as against experimental data. Also users can write own templates and use it on layouts page. Each plot on the page can be viewed in static and interactive mode. In case of two or more versions selected one can plot ratio.

Страница Stat comparison предназначена для статистической валидации выбранного теста и статистического сравнения двух версий G4 или одной версии и экспериментальных данных. Для каждой пары плотов пользователю выводятся значения $\chi^2/n.d.f.$, Kolmogorov-Smirnov test, а также $p$-value. Значения $\chi^2$ и Kolmogorov-Smirnov test вычисляются на стороне клиента в фоновых процессах браузера используя WebWorkers API, WebWorkers JavaScript code выдает те же результаты, что и ROOT. Имеется возможность сортировки таблицы $\chi^2$ по значения параметров теста и значениям $\chi^2$.

На странице Exp data приводится таблица всех экспериментальных данных, загруженных в базу данных приложения.

На странице Lookup tables представлены значения табличных значений приложения (частицы, наблюдаемые величины, версии G4 и т.д.).

Каждый плот, представленный в приложении, имеет возможность интерактивного просмотра с использованием JSROOT\cite{JSROOT} и возможность скачивания в PNG, ROOT, EPS. Также для пользователя доступны данные в форматах JSON и Gnuplot. 

На сайте implemented аутентификация пользователей, успешное прохождение которой открывает доступ к сравнению тестовых версий Geant4.

%Don't forget to give each section, subsection, subsubsection, and
%paragraph a unique label (see Sect.~\ref{sec-1}).

%For one-column wide figures use syntax of figure~\ref{fig-1}
%\begin{figure}[h]
% Use the relevant command for your figure-insertion program
% to insert the figure file.
%\centering
%\includegraphics[width=1cm,clip]{tiger}
%\caption{Please write your figure caption here}
%\label{fig-1}       % Give a unique label
%\end{figure}

%For two-column wide figures use syntax of figure~\ref{fig-2}
%\begin{figure*}
%\centering
% Use the relevant command for your figure-insertion program
% to insert the figure file. See example above.
% If not, use
%\vspace*{5cm}       % Give the correct figure height in cm
%\caption{Please write your figure caption here}
%\label{fig-2}       % Give a unique label
%\end{figure*}

%For figure with sidecaption legend use syntax of figure
%\begin{figure}
% Use the relevant command for your figure-insertion program
% to insert the figure file.
%\centering
%\sidecaption
%\includegraphics[width=5cm,clip]{tiger}
%\caption{Please write your figure caption here}
%\label{fig-3}       % Give a unique label
%\end{figure}

%For tables use syntax in table~\ref{tab-1}.
%\begin{table}
%\centering
%\caption{Please write your table caption here}
%\label{tab-1}       % Give a unique label
% For LaTeX tables you can use
%\begin{tabular}{lll}
%\hline
%first & second & third  \\\hline
%number & number & number \\
%number & number & number \\\hline
%\end{tabular}
% Or use
%\vspace*{5cm}  % with the correct table height
%\end{table}