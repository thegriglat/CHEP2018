\section{Web application}
\label{sec-webapplication}

\subsection{Server side/Back end}
\label{sec-webapplication-server}

The server is the core of the geant-val system. It provides Web API for the clients to request data from PostgreSQL database
and generates high quality plots on the fly when they are requested.

Серверная часть приложения написана на JavaScript на платформе Node.js с использование PostgreSQL в качестве базы данных.

Для генерации плотов используется написанная нами ROOT-based С++ утилита {\tt plotter}. Для пользователя плоты доступны в форматах PNG, ROOT, EPS, JSON, Gnuplot.

\subsection{Client side/Front end}
\label{sec-webapplication-client}

Клиентская часть приложения написана на JavaScript с использованием фреймворка AngularJS версии 1.4.8. Сайт состоит из 4 основных разделов:
\begin{itemize}
    \item Layouts
    \item Stat comparison
    \item Experimental data
    \item Lookup tables
\end{itemize}

Страница Layouts предназначена для визуального сравнения предустановленных набора графиков для G4 и экспериментальных данных. Пользователь имеет возможность использовать уже существующие теплейты или загрузить свой. Каждый график в темплейте имеет возможность статического и интерактивного просмотра, а также построение ratio plots.

Страница Stat comparison предназначена для статистической валидации выбранного теста и статистического сравнения двух версий G4 или одной версии и экспериментальных данных. Для каждой пары плотов пользователю выводятся значения $\chi^2/n.d.f.$, Kolmogorov-Smirnov test, а также $p$-value. Значения $\chi^2$ и Kolmogorov-Smirnov test вычисляются на стороне клиента в фоновых процессах браузера используя WebWorkers API, WebWorkers JavaScript code выдает те же результаты, что и ROOT. Имеется возможность сортировки таблицы $\chi^2$ по значения параметров теста и значениям $\chi^2$.

На странице Exp data приводится таблица всех экспериментальных данных, загруженных в базу данных приложения.

На странице Lookup tables представлены значения табличных значений приложения (частицы, наблюдаемые величины, версии G4 и т.д.).

Каждый плот, представленный в приложении, имеет возможность интерактивного просмотра с использованием JSROOT\cite{JSROOT} и возможность скачивания в PNG, ROOT, EPS. Также для пользователя доступны данные в форматах JSON и Gnuplot. 

На сайте implemented аутентификация пользователей, успешное прохождение которой открывает доступ к сравнению тестовых версий Geant4.

%Don't forget to give each section, subsection, subsubsection, and
%paragraph a unique label (see Sect.~\ref{sec-1}).

%For one-column wide figures use syntax of figure~\ref{fig-1}
%\begin{figure}[h]
% Use the relevant command for your figure-insertion program
% to insert the figure file.
%\centering
%\includegraphics[width=1cm,clip]{tiger}
%\caption{Please write your figure caption here}
%\label{fig-1}       % Give a unique label
%\end{figure}

%For two-column wide figures use syntax of figure~\ref{fig-2}
%\begin{figure*}
%\centering
% Use the relevant command for your figure-insertion program
% to insert the figure file. See example above.
% If not, use
%\vspace*{5cm}       % Give the correct figure height in cm
%\caption{Please write your figure caption here}
%\label{fig-2}       % Give a unique label
%\end{figure*}

%For figure with sidecaption legend use syntax of figure
%\begin{figure}
% Use the relevant command for your figure-insertion program
% to insert the figure file.
%\centering
%\sidecaption
%\includegraphics[width=5cm,clip]{tiger}
%\caption{Please write your figure caption here}
%\label{fig-3}       % Give a unique label
%\end{figure}

%For tables use syntax in table~\ref{tab-1}.
%\begin{table}
%\centering
%\caption{Please write your table caption here}
%\label{tab-1}       % Give a unique label
% For LaTeX tables you can use
%\begin{tabular}{lll}
%\hline
%first & second & third  \\\hline
%number & number & number \\
%number & number & number \\\hline
%\end{tabular}
% Or use
%\vspace*{5cm}  % with the correct table height
%\end{table}