\section{Introduction}
\label{sec-introduction}
Geant4~\cite{Geant4} is a toolkit for the simulation of the passage of particles through matter. Its areas of application include high energy, nuclear and accelerator physics, as well as studies in medical and space science.
The Geant4 team produces a public release once a year, with monthly internal testing releases. These results need to be validated.

Physics validation of Geant4, which includes regression testing, comparison of performance between different physics models and comparison with corresponding experimental measurements, requires analysing a large number of histograms produced by Geant4 tests. %These tests were developed by Geant4 community, and utilise various approaches and a wide stack of technologies, such as PHP, C++, Python and others. This diversity makes it very hard to quickly perform full validation of release, as it requires running all developer's micro frameworks and merging data into one view. In addition, different formats of experimental data are often used, and that prevents sharing the same data between tests.

A tool for release validation was developed to perform statistical and visual comparisons with respect to previous release(s) and to the experimental data.

%The presented tool is a website working with a special framework designed for Geant4 tests configuration and execution.
