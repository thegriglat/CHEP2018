\begin{thebibliography}{}
%
% and use \bibitem to create references.
%
\bibitem{Geant4}
S.~Agostinelli {\it et al.} [GEANT4 Collaboration],
  %``GEANT4: A Simulation toolkit,''
  Nucl.\ Instrum.\ Meth.\ A {\bf 506}, 250 (2003).
  doi:10.1016/S0168-9002(03)01368-8
  %%CITATION = doi:10.1016/S0168-9002(03)01368-8;%%

\bibitem{AngularJS}
AngularJS, v.1.4.8, 20.11.2015, \url{https://github.com/angular/angular.js/commit/75e876424da5f569481488d03cf3a61441341513}

\bibitem{Bootstrap}
Bootstrap, v.3.3.7, 25.07.2016, \url{https://github.com/twbs/bootstrap/commit/0b9c4a4007c44201dce9a6cc1a38407005c26c86}

\bibitem{KaTeX}
KaTeX, v.0.8.3, 28.08.2017,
\url{https://github.com/Khan/KaTeX/commit/dd46d1b00be67b9c859786e134c35495b21ca66b}

\bibitem{JSROOT}
Bertrand Bellenot and Sergey Linev 2015, J. Phys.: Conf. Ser.,664,  062033

\bibitem{ROOT}
    Rene Brun and Fons Rademakers,
    ROOT - An Object Oriented Data Analysis Framework,
    Proceedings AIHENP'96 Workshop, Lausanne, Sep. 1996, Nucl. Inst. \& Meth. in Phys. Res. A 389 (1997) 81-86. See also (\url{http://root.cern.ch}).

%\bibitem{RefJ}
% Format for Journal Reference
%Journal Author, Journal \textbf{Volume}, page numbers (year)
% Format for books
%\bibitem{RefB}
%Book Author, \textit{Book title} (Publisher, place, year) page numbers
% etc
\end{thebibliography}